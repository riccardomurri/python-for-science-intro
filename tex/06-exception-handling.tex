\documentclass[english,serif,mathserif,xcolor=pdftex,dvipsnames,table]{beamer}

\usepackage[T1]{fontenc}
\usepackage[utf8]{inputenc}

\usetheme[informal]{s3it}
\usepackage{s3it}

\title[5. Exception handling]{%
  A Short and Incomplete Introduction to Python
}
\subtitle{\bfseries Part 6: Exception handling}
\author[R.~Murri]{%
  \textbf{Riccardo Murri} \texttt{<riccardo.murri@uzh.ch>}, \\
  Sergio Maffioletti \texttt{<sergio.maffioletti@uzh.ch>}
  \\
  S3IT: Services and Support for Science IT,
  \\
  University of Zurich
}
\date{June 7--8, 2018}


\begin{document}

% title frame
\maketitle


\part{Exception handling}

\begin{frame}[fragile]
  \frametitle{Exceptions}

  ``Exceptions'' is the name given in Python to error conditions.

  \+
  You can write code that intercepts some error conditions and
  reacts appropriately.

  \+
  \begin{seealso}
    \url{http://docs.python.org/library/exceptions.html}
  \end{seealso}
\end{frame}


\begin{frame}[fragile]
  \frametitle{What does an exception look like?}
\begin{lstlisting}
>>> stream.write('foo')
Traceback (most recent call last):
  File "<stdin>", line 1, in <module>
IOError: ~\ttfamily File not open for writing~
\end{lstlisting}
\end{frame}


\begin{frame}[fragile]
  \frametitle{What does an exception look like?}
\begin{lstlisting}
>>> stream.write('foo')
Traceback (most recent call last):
  File "<stdin>", line 1, in <module>
IOError: ~\HL{File not open for writing}~
\end{lstlisting}

  \+
  This is the exception \emph{message}: it is supposed to be read
  by the (human) user.
\end{frame}


\begin{frame}[fragile]
  \frametitle{What does an exception look like?}
\begin{lstlisting}
>>> stream.write('foo')
Traceback (most recent call last):
  File "<stdin>", line 1, in <module>
~\HL{IOError}~: File not open for writing
\end{lstlisting}

  \+ This is the exception \emph{class name}; it is used for catching
  exceptions (syntax in the next slide).
\end{frame}


\begin{frame}[fragile]
\begin{lstlisting}
try:
  # code that might raise an exception
except SomeException:
  # handle some exception
except AnotherException as ex:
  # the actual Exception instance
  # is available as variable `ex'
finally:
  # performed on exit in any case
\end{lstlisting}

  \+
  The optional \lstinline|finally| clause is executed on exit from the
  \lstinline|try| or \lstinline|except| block in \emph{any} case.

  \begin{references}
    \scriptsize
    \url{http://docs.python.org/reference/compound_stmts.html#try}
\end{references}
\end{frame}


\begin{frame}[fragile]
  \frametitle{Raising exceptions in your code}
  Use the \lstinline|raise| statement with an \texttt{Exception}
  instance:
\begin{lstlisting}
if an_error_occurred:
  raise RuntimeError("Spider sense is tingling.")
\end{lstlisting}
\end{frame}


\begin{frame}
  \begin{exercise*}[6.A]
    Try loading file \texttt{values2.txt} with the \lstinline`load_data()`
    function from Exercise~4.A -- what exception does Python raise?

    \+
    Edit the \lstinline`load_data()` function into a \lstinline`load_data2()`
    that \emph{ignores} any line that does not contain an integer number.

    \+
    Bonus points if you can write \lstinline`load_data2()` so that it has
    exactly the same output of \lstinline`load_data()`, i.e. minimize the number
    of rejected input lines.
  \end{exercise*}
\end{frame}


\end{document}

%%% Local Variables:
%%% mode: latex
%%% TeX-master: t
%%% End:
