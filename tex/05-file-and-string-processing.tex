\documentclass[english,serif,mathserif,xcolor=pdftex,dvipsnames,table]{beamer}

\usepackage[T1]{fontenc}
\usepackage[utf8]{inputenc}

\usetheme[informal]{s3it}
\usepackage{s3it}

\title[4. Files and strings]{%
  A Short and Incomplete Introduction to Python
}
\subtitle{\bfseries Part 5: File I/O and string processing}
\author[R.~Murri]{%
  \textbf{Riccardo Murri} \texttt{<riccardo.murri@uzh.ch>}, \\
  Sergio Maffioletti \texttt{<sergio.maffioletti@uzh.ch>}
  \\
  S3IT: Services and Support for Science IT,
  \\
  University of Zurich
}
\date{January 21, 2019}


\begin{document}

% title frame
\maketitle


\part{Strings}


\begin{frame}[fragile]
  In Python, Strings are sequences of characters:
  \begin{itemize}
  \item They can be indexed and sliced like \lstinline|list|'s and other sequences:
\begin{semiverbatim}\small
\In s = 'python'
\In s[:2]
\Out 'py'
\end{semiverbatim}
  \item They are \textbf{homogeneous}: items of a string are always characters.
  \item They are \textbf{immutable}: you can only alter a string through functions that make a (modified) copy:
\begin{semiverbatim}\small
\In s[0] = 'c'
...
TypeError: 'str' object does not support
item assignment
\end{semiverbatim}
  \end{itemize}

\end{frame}


\begin{frame}[fragile]
  \frametitle{Operations on strings, I}
  \begin{describe}{%
      \lstinline|s.capitalize()|,
      \lstinline|s.lower()|,
      \lstinline|s.upper()|}
    Return a \emph{copy} of the string capitalized / turned all lowercase /
    turned all uppercase.
  \end{describe}

  \begin{describe}{\lstinline|s.split(t)|}
    Split \texttt{s} at every occurrence of \texttt{t} and return a list
    of parts.  If \texttt{t} is omitted, split on whitespace.
  \end{describe}

  \begin{describe}{\lstinline|s.startswith(t)|,
      \lstinline|s.endswith(t)|}
    Return \texttt{True} if \texttt{t} is the initial/final substring
    of \texttt{s}.
  \end{describe}

  \begin{references}
    \url{http://docs.python.org/library/stdtypes.html#string-methods}
  \end{references}
\end{frame}


\begin{frame}[fragile]
  \frametitle{Operations on strings, II}
  \begin{describe}{\lstinline|s.replace(old, new)|}
    Return a \emph{copy} of string \texttt{s} with all occurrences of
    substring \texttt{old} replaced by \texttt{new}.
  \end{describe}

  \begin{describe}{%
      \lstinline|s.lstrip()|,
      \lstinline|s.rstrip()|,
      \lstinline|s.strip()|}
    Return a \emph{copy} of the string with the leading (resp.\ trailing,
    resp.\ leading \emph{and} trailing) whitespace removed.
  \end{describe}

  \begin{references}
    \url{http://docs.python.org/library/stdtypes.html#string-methods}
  \end{references}
\end{frame}


\begin{frame}[fragile]
  \small

  \begin{exercise*}[5.A]
    Write a function \lstinline|split_comma(s)| which, given a string
    \lstinline|s| (containing comma-separated items) returns a
    \emph{list} of the items.  For example:
\begin{semiverbatim}
\In split_comma("a,b,c")
\Out ['a', 'b', 'c']
\end{semiverbatim}
  \end{exercise*}

  \begin{exercise*}[5.B]
    Modify \lstinline|split_comma| to remove whitespace around the
    returned items, so that \lstinline|split_comma("a, b, c")| and
    \lstinline|split_comma("a,b,c")| return the same result
    \lstinline|['a', 'b', 'c']|.
  \end{exercise*}

  \+
  \begin{exercise*}[5.C]
    Write a function \lstinline|unquotes(s)| which, given a string \lstinline|s| returns a copy of \lstinline|s| with:
    %\begin{inparaenum}
    {\color{gray}\em 1.} All leading and trailing whitespace removed,
    {\color{gray}\em 2.} Initial and final double quote ``\lstinline|"|'' characters removed (if any).
    %\end{inparaenum}
    For example:
\begin{semiverbatim}
\In unquote(' "abc"')
\Out 'abc'
\end{semiverbatim}
  \end{exercise*}
\end{frame}


\begin{frame}
  \frametitle{Modifying strings}
  Python strings are \emph{immutable} so direct modification is not allowed.

  \+
  There are two options to modify strings:
  \begin{itemize}
  \item Use built-in operations (just seen)
  \item Convert to list (which is mutable) and back to string
  \end{itemize}
\end{frame}


\begin{frame}[fragile,fragile]
  \frametitle{String to list and back}
  The \texttt{list()} constructor will convert a string into a list of
  its constituent characters:
\begin{semiverbatim}
\In s = 'uzh'
\In l = list(s)
\In print(l)
\Out ['u', 'z', 'h']
\end{semiverbatim}

  \+
  Conversely, \texttt{s.join(L)} will join all strings in
  \texttt{L}, interposing occurences of \texttt{s}:
\begin{semiverbatim}
\In '-'.join(l)
\Out 'u-z-h'
\In ''.join(l)  # empty string as separator
\Out 'uzh'
\end{semiverbatim}
\end{frame}


\part{File I/O}

\begin{frame}[fragile]
  \frametitle{File I/O}

  Code for processing a text file usually looks like this:
\begin{lstlisting}
with open(filename, 'r') as stream:
  # prepare for processing
  for line in stream:
    # process each line
\end{lstlisting}
\end{frame}


\begin{frame}[fragile]
  \frametitle{File I/O}

\begin{lstlisting}
with ~\HL{open(filename, 'r')}~ as stream:
  # prepare for processing
  for line in stream:
    # process each line
\end{lstlisting}

  \+ The \lstinline|open(path, mode)| function opens the file located at
  \texttt{path} and returns a ``file object'' that can be used for reading
  and/or writing.

  \+ Mode is one of '\texttt{r}', '\texttt{w}' or '\texttt{a}' for reading,
  overwriting (truncates on open), appending. Details two slides forward!
\end{frame}


\begin{frame}[fragile]
  \frametitle{File I/O}

\begin{lstlisting}
~\HL{with}~ open(filename, 'r') ~\HL{as}~ stream:
  # prepare for processing
  for line in stream:
    # process each line
\end{lstlisting}

  \+
  This is equivalent to \lstinline|stream = open(...)| but in addition
  \emph{closes} the file when the code in the \texttt{with}-block is done.

  \+
  There are many more uses of the \texttt{with} statement besides automatically
  closing files, check out \url{https://jeffknupp.com/blog/2016/03/07/python-with-context-managers/}
\end{frame}


\begin{frame}[fragile]
  \frametitle{File I/O}

\begin{lstlisting}
with open(filename, 'r') as stream:
  # prepare for processing
  ~\HL{for line in stream}~:
    # process each line
\end{lstlisting}

  \+ A \texttt{for}-loop can be used to process all lines in a file, as if the
  file were a list.
\end{frame}


\begin{frame}
  \frametitle{File open modes}
  \begin{center}
    \begin{tabular}{>{\tt}rl}
      {\rm\em ~~~~~First char.}
      & {\rm\em Open file for \ldots}
      \\
      \texttt{r}
      & reading
      \\
      \texttt{w}
      & writing, position at beginning of file
      \\
      \texttt{a}
      & writing, position at end of file (append)
      \\
      \texttt{x}
      & writing, create new file (error if exists)
      \\
    \end{tabular}

    \+
    \begin{tabular}{>{\tt}rl}
      {\rm\em More char.'s}
      & {\rm\em Open file for \ldots}
      \\
      \texttt{+}
      & reading \emph{and} writing
      \\
      \texttt{b}
      & use \texttt{bytes} for I/O operations
      \\
      \texttt{t}
      & use \texttt{str} for I/O operations (default)
      \\
    \end{tabular}

    \+ Run \texttt{help(open)} to get all details.
  \end{center}
\end{frame}

\begin{frame}[fragile]
  \frametitle{More on File I/O}

  The \lstinline|.read()| method can be used to read the \emph{whole} contents
  of a file in one go as a single string:
\begin{lstlisting}
>>> s = stream.read()
\end{lstlisting}

  \+
  Method \lstinline|.readlines()| returns a list of all lines in the file:
\begin{lstlisting}
>>> L = stream.readlines()
\end{lstlisting}

  \begin{references}
    \url{http://docs.python.org/library/stdtypes.html#file-objects}
  \end{references}
\end{frame}


\begin{frame}[fragile,label=typeconv]
  \frametitle{Type conversions}
  \begin{description}
  \item[str($x$)] Converts the argument $x$ to a string; for numbers,
    the base 10 representation is used.
  \item[int($x$)] Converts its argument $x$ (a number or a string) to an integer;
    if $x$ is a a floating-point literal, decimal digits are truncated.
  \item[float($x$)] Converts its argument $x$ (a number or a string) to a
    floating-point number.
  \end{description}
\end{frame}


\begin{frame}[fragile]
  \frametitle{The `{\ttfamily\bfseries in}' operator (1)}

  Use the \lstinline|in| operator to test for presence of an item in a
  collection.

  \begin{describe}{\lstinline|x in S|}
    Evaluates to \texttt{True} if \lstinline|x| is equal to a \emph{value}
    contained in the \lstinline|S| sequence (list, tuple, set).
  \end{describe}

  \begin{describe}{\lstinline|S in T|}
    Evaluates to \texttt{True} if \lstinline|S| is a substring of
    string \lstinline|T|.
  \end{describe}

\end{frame}


\begin{frame}[fragile]
  \begin{exercise*}[5.D]
    Write a function \lstinline|load_data(filename)| that reads a file
    containing one integer number per line, and return a list of the
    integer values.

    \+
    Test it with the
    \href{https://raw.github.com/gc3-uzh-ch/python-course/master/values.txt}{values.txt}
    file:
\begin{lstlisting}
>>> load_data('values.dat')
[299850, 299740, 299900, 300070, 299930]
\end{lstlisting}
  \end{exercise*}

  \begin{exercise*}[5.E]
    Write a function \lstinline|fgrep(pattern, filename)| which returns a list of
    all lines in file \texttt{filename} which contain string \texttt{pattern}.
  \end{exercise*}
\end{frame}


\begin{frame}[fragile]
  \frametitle{Filesystem operations, I}
  \small
  These functions are available from the \texttt{os} module.

  \begin{describe}{\lstinline|os.getcwd()|, \lstinline|os.chdir(path)|}
    Return the path to the current working directory /
    Change the current working directory to \texttt{path}.
  \end{describe}

  \begin{describe}{\lstinline|os.listdir(dir)|}
    Return list of entries in directory \texttt{dir} (omitting
    `\texttt{.}' and `\texttt{..}')
  \end{describe}

  % \begin{describe}{\lstinline|os.mkdir(path)|}
  %   Create a directory; fails if the directory already exists.
  %   Assumes that all parent directories exist already.
  % \end{describe}

  \begin{describe}{\lstinline|os.makedirs(path)|}
    Create a directory; no-op if the directory already exists.
    Creates all the intermediate-level directories needed to contain
    the leaf.
  \end{describe}

  \begin{describe}{\lstinline|os.rename(old,new)|}
    Rename a file or directory from \texttt{old} to \texttt{new}.
  \end{describe}

  \begin{references}
    \url{http://docs.python.org/library/os.html}
  \end{references}
\end{frame}


\begin{frame}[fragile]
  \frametitle{Filesystem operations, II}
  These functions are available from the \texttt{os.path} module.

  \begin{describe}{\lstinline|os.path.exists(path)|, \lstinline|os.path.isdir(path)|, \lstinline|os.path.isfile(path)|}
    Return \texttt{True} if \texttt{path} exists / is a directory / is
    a regular file.
  \end{describe}

  \begin{describe}{\lstinline|os.path.basename(path)|,
      \lstinline|os.path.dirname(path)|}
    Return the base name (the part after the last `\texttt{/}'
    character) or the directory name (the part before the last
    \texttt{/} character).
  \end{describe}

  \begin{describe}{\lstinline|os.path.abspath(path)|}
    Make \texttt{path} absolute (i.e., start with a \texttt{/}).
  \end{describe}

  \begin{references}
    \url{http://docs.python.org/library/os.path.html}
  \end{references}
\end{frame}


\end{document}

%%% Local Variables:
%%% mode: latex
%%% TeX-master: t
%%% End:
