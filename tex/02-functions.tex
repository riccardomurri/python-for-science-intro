\documentclass[english,serif,mathserif,xcolor=pdftex,dvipsnames,table]{beamer}

\usepackage[T1]{fontenc}
\usepackage[utf8]{inputenc}

\usetheme[informal]{s3it}
\usepackage{s3it}

\title[2. Functions]{%
  A Short and Incomplete Introduction to Python
}
\subtitle{\bfseries Part 2: Functions}
\author[R.~Murri]{%
  Sergio Maffioletti \texttt{<riccardo.murri@uzh.ch>}
  \\
  Riccardo Murri \texttt{<riccardo.murri@uzh.ch>}
  \\
  S3IT: Services and Support for Science IT,
  \\
  University of Zurich
}
\date{September, 2017}


%% simulate IPython prompt
\newcounter{prompt}
\newcommand\I{%
  \global\def\promptnr{\arabic{prompt}}%
  \stepcounter{prompt}%
  {\color{blue}\bfseries In~[\promptnr]:}%
}
\renewcommand\O{{\color{red}\bfseries Out[\promptnr]:}}
\newcommand\C{{\color{blue}   ...:}}

\begin{document}

% title frame
\maketitle


\part{Functions}

\begin{frame}[fragile,label=func1]
  \frametitle{Functions, I}
  Functions are called by postfixing the function name with a
  parenthesized argument list.

  \+
\begin{lstlisting}
>>> int("42")
42
>>> int(4.2)
4
>>> float(42)
42.0
>>> str(42)
'42'
>>> str()
''
\end{lstlisting}
\end{frame}


\begin{frame}[fragile]
  \frametitle{Functions, II}
  Some functions can take a variable number of arguments. For instance:

  \+
  \begin{description}
    \item[sum($x_0$, \ldots, $x_n$)] Return $x_0 + \cdots + x_n$.
    \item[max($x_0$, \ldots, $x_n$)] Return the maximum of $\{ x_0, \ldots, x_n \}$
    \item[min($x_0$, \ldots, $x_n$)] Return the minimum of $\{ x_0, \ldots, x_n \}$
  \end{description}

  \+
  Examples:
\begin{semiverbatim}
\I min(1,2,3)
\O 1
\I max(1,2)
\O 2
\end{semiverbatim}
\end{frame}


\begin{frame}[fragile]
  \frametitle{The most important function of all}
  \begin{description}
  \item[help(\texttt{fn})] Display help on the function named \texttt{fn}
  \end{description}

  \+
  \begin{exercise*}[2.A]
    What happens if you type these at the prompt?
    \begin{itemize}
    \item \texttt{help(abs)}
    \item \texttt{help(help)}
    \end{itemize}
  \end{exercise*}
\end{frame}


\begin{frame}[fragile]
  \frametitle{How to define new functions}
  \begin{columns}[t]
    \begin{column}{0.5\textwidth}
\begin{lstlisting}
~\HL{\textbf{def} greet(name):}~
  """
  A friendly function.
  """
  print ("Hello, " + name + "!")

# the customary greeting
greet("world")
\end{lstlisting}
    \end{column}
    \begin{column}{0.5\textwidth}
      \raggedleft
      The \textbf{def} statement starts a function definition.
    \end{column}
  \end{columns}
\end{frame}

\begin{frame}[fragile]
  \begin{columns}[t]
    \begin{column}{0.5\textwidth}
\begin{lstlisting}
def greet(name):
  """
  A friendly function.
  """
  ~\HL{print ("Hello, " + name + "!")}~

# the customary greeting
greet("world")
\end{lstlisting}
    \end{column}
    \begin{column}{0.5\textwidth}
      \raggedleft
      \textbf{Indentation is significant in Python}: it is used to delimit
      blocks of code, like `\texttt{\{}' and `\texttt{\}}' in Java and C.
    \end{column}
  \end{columns}
\end{frame}

\begin{frame}[fragile]
  \begin{columns}[t]
    \begin{column}{0.5\textwidth}
\begin{lstlisting}
def greet(name):
  """
  A friendly function.
  """
  print ("Hello, " + name + "!")

~\HL{\it\tt\#\ the customary greeting}~
greet("world")
\end{lstlisting}
    \end{column}
    \begin{column}{0.5\textwidth}
      \raggedleft
      (This is a comment. It is ignored by Python, just like blank lines.)
    \end{column}
  \end{columns}
\end{frame}

\begin{frame}[fragile]
  \begin{columns}[t]
    \begin{column}{0.5\textwidth}
\begin{lstlisting}
def greet(name):
  """
  A friendly function.
  """
  print ("Hello, " + name + "!")

# the customary greeting
~\HL{greet("world")}~
\end{lstlisting}
    \end{column}
    \begin{column}{0.5\textwidth}
      \raggedleft
      This calls the function just defined.
    \end{column}
  \end{columns}
\end{frame}

\begin{frame}[fragile]
  \begin{columns}[t]
    \begin{column}{0.5\textwidth}
\begin{lstlisting}
def greet(name):
  ~\HL{"""}~
  ~\HL{A friendly function.}~
  ~\HL{"""}~
  print ("Hello, " + name + "!")

# the customary greeting
greet("world")
\end{lstlisting}
    \end{column}
    \begin{column}{0.5\textwidth}
      \raggedleft
      What is this? The answer in the next exercise!
    \end{column}
  \end{columns}
\end{frame}

\begin{frame}
  \begin{exercise*}[2.B]
    Type and run the code on the previous page at the interactive
    prompt. (Pay attention to indentation!)

    \+
    What's the result of evaluating the function \texttt{greet("world")}?

    \+
    What does \texttt{help(greet)} output?
  \end{exercise*}
\end{frame}


\begin{frame}[fragile]
  \frametitle{Default values}

  Function arguments can have default values.
\begin{lstlisting}
>>> def hello(name='world'):
...   print ("Hello, " + name)
...
>>> hello()
'Hello, world'
\end{lstlisting}
\end{frame}


\begin{frame}[fragile]
  \frametitle{Named arguments}
Python allows calling a function with named arguments:
\begin{lstlisting}
hello(name="Alice")
\end{lstlisting}

\+
When passing arguments by name, they can be passed in any order:
\begin{lstlisting}
>>> from fractions import Fraction
>>> Fraction(numerator=1, denominator=2)
Fraction(1, 2)
>>> Fraction(denominator=2, numerator=1)
Fraction(1, 2)
\end{lstlisting}
\end{frame}

\begin{frame}[fragile]
  \frametitle{The `return` statement}

  \begin{columns}
    \begin{column}{0.5\textwidth}
      \begin{lstlisting}
def double(x):
  ~\HL{return x+x}~

double(3) == 6
      \end{lstlisting}
    \end{column}
    \begin{column}{0.5\textwidth}
      \raggedleft The result of a function evaluation is set by the
      \textit{return} statement.

     \+
      If no \texttt{return} is present, the function returns the
      special value \texttt{None}.

         \end{column}
  \end{columns}
\+
  \begin{columns}
    \begin{column}{0.5\textwidth}
      \begin{lstlisting}
def double(x):
  return x+x
  # the following line
  # is never exec'd
  ~\HL{print('Hello')}~
      \end{lstlisting}
    \end{column}
    \begin{column}{0.5\textwidth}
      \raggedleft After executing \texttt{return} the control flow
      leaves the function.
    \end{column}
  \end{columns}

\end{frame}


\part{Basic control flow}
\begin{frame}[fragile]
  \frametitle{Conditionals}
  Conditional execution uses the \texttt{if} statement:
\begin{lstlisting}
if ~\it expr~:
  # indented block
elif ~\it other-expr~:
  # indented block
else:
  # executed if none of the above matched
\end{lstlisting}

  \+The \texttt{elif} can be repeated, with different conditions, or
  left out entirely.

  \+
  Also the \texttt{else} clause is optional.

  \+
  \begin{question}
    Where's the `end if'?

    \pause{There's no `end if': indentation delimits blocks!}
  \end{question}
\end{frame}


\begin{frame}[fragile]
  \frametitle{\texttt{while}-Loops}
  Conditional looping uses the \texttt{while} statement:
\begin{lstlisting}
while ~\it expr~:
  # indented block
\end{lstlisting}
% else:
%   # executed at natural end of the loop

  \+
  To break out of a \texttt{while} loop, use the \texttt{break}
  statement.

  \+
  Use the \texttt{continue} statement anywhere in the indented
  block to jump back to the \texttt{while} statement.

  % \+
  % If a loop is exited via a \texttt{break} statement, the
  % \texttt{else} clause is \emph{not} executed.

  % \+
  % The \texttt{else} clause is optional.
\end{frame}


\begin{frame}
  \begin{exercise*}[2.C]
    Modify the \texttt{greet()} function to print ``Welcome back!'' if the
    argument \texttt{name} is the string \texttt{'Python'}.
  \end{exercise*}
\end{frame}


\part{Modules}

\begin{frame}[fragile]
  \frametitle{Modules, I}
  The \texttt{import} statement reads a \texttt{.py} file, executes
  it, and makes its contents available to the current program.
\begin{lstlisting}
>>> import hello
Hello, world!
\end{lstlisting}

  \+
  \textbf{Modules are only read once}, no matter how many times an
  \texttt{import} statement is issued.
\begin{lstlisting}
>>> import hello
Hello, world!
>>> import hello
>>> import hello
\end{lstlisting}

\end{frame}


\begin{frame}[fragile]
  \frametitle{Modules, II}
  Modules are \emph{namespaces:} functions and variables defined in
  a module must be prefixed with the module name when used in other
  modules:
\begin{lstlisting}
>>> hello.greet("Python")
Hello, Python!
\end{lstlisting}

  \+
  To import definitions into the current namespace, use the
  `\texttt{from $x$ import $y$}' form:
\begin{lstlisting}
>>> from hello import greet
>>> greet("Python")
Hello, Python!
\end{lstlisting}
\end{frame}


\end{document}

%%% Local Variables:
%%% mode: latex
%%% TeX-master: t
%%% End:
