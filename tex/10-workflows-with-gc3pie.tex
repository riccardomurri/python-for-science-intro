\documentclass[english,serif,mathserif,xcolor=pdftex,dvipsnames,table]{beamer}

\usepackage[T1]{fontenc}
\usepackage[utf8]{inputenc}

\usetheme[informal]{s3it}
\usepackage{s3it}

\title[10. Workflows]{%
  Introduction to workflows
  using GC3Pie
}
\author[R.~Murri]{%
  \textbf{Riccardo Murri} \texttt{<riccardo.murri@uzh.ch>}, \\
  Sergio Maffioletti \texttt{<sergio.maffioletti@uzh.ch>}
  \\
  {\em S3IT: Services and Support for Science IT}
  \\
  University of Zurich
}
\date{January 2019}


\begin{document}

% title frame
\maketitle

\begin{frame}
  \frametitle{What is GC3Pie?}
  GC3Pie is \ldots
  \begin{enumerate}
  \item An \emph{opinionated} Python framework for defining and running computational workflows;
  \item \alert<2>{A \emph{rapid development toolkit} for running user applications on clusters and IaaS cloud resources;}
  \item The worst name ever given to a middleware piece\ldots
  \end{enumerate}

  \+
  As \emph{users}, \alert<2>{you're mostly interested in this part.}
\end{frame}


\part{Uses of GC3Pie}

\begin{frame}[fragile]
  \frametitle{Uses of GC3Pie: parameter sweep}

  You have a simulation code that is dependent on a number of parameters.

  \+
  Run the code for all possible combinations of parameters.

  \+
  Then collect all the outputs and post-process to get a
  statistical overview.
\end{frame}


\begin{frame}[fragile]
  \frametitle{Uses of GC3Pie: model calibration}

  You have a simulation code that is dependent on a number of parameters.

  \+
  Run the code for all possible combinations of parameters, and
  find the ones that ``best'' approximate a given experimental result.
\end{frame}


\begin{frame}[fragile]
  \frametitle{Uses of GC3Pie: parallel processing}

  Run the same program over and over again,
  feeding it different input files each time.

  \+
  Then collect all the outputs and post-process to get a
  statistical overview.
\end{frame}


\begin{frame}[fragile]
  \frametitle{Uses of GC3Pie: parallel processing}
  (At times, you chop a large input file into pieces and process each one separately instead.)

  \+
  \begin{quote}
    ``For example, say we have a de novo assembly of 100,000
    contigs. If we run 1 BLAST job against NR it could take as long as
    50,000 minutes/35 days!! (30sec/query sequence), however if we
    split this job into subsets of 5,000 sequences and ran 20 jobs in
    "parallel" on a cluster, our total run-time is reduced to only 41
    hours.''
  \end{quote}
  \begin{references}
    \url{http://sfg.stanford.edu/BLAST.html}
  \end{references}
\end{frame}


\begin{frame}
  \frametitle{Uses of GC3Pie: workflows}

  \begin{center}
    Orchestrate execution of several applications:
    some steps may run in parallel, some might need to be sequenced.

    \+
    \only<1>{%
      \begin{tabular}[c]{ccc}
        \includegraphics[width=0.4\textwidth]{fig/butterfly.jpg}
        &
          \includegraphics[width=0.1\textwidth,totalheight=0.25\textheight]{fig/arrow.pdf}
        &
          \includegraphics[width=0.4\textwidth]{fig/warholized-butterfly.jpg}
      \end{tabular}

      \+
      {\tiny\em (Butterfly image originally from
        \href{https://pixabay.com/it/farfalla-rondine-farfalla-a-coda-1228639/}{pixabay.com})}
    }%
    \only<2>{%
      \includegraphics[width=0.70\textwidth]{fig/warholize-wkf-with-results.pdf}
    }%
  \end{center}
\end{frame}


\part{How GC3Pie works}

\begin{frame}
  \frametitle{A typical high-throughput script structure}

  \begin{enumerate}
  \item Initialize computational resources
  \item Prepare programs and inputs for submission
  \item Submit tasks
  \item Monitor task status (loop)
  \item Retrieve results
  \item Postprocess and display
  \end{enumerate}
\end{frame}


\fullgraphicsframe{fig/gc3pie-running-1.pdf}

\fullgraphicsframe{fig/gc3pie-running-3.pdf}

%\fullgraphicsframe{fig/gc3pie-running-2.pdf}


\begin{frame}
  \frametitle{What GC3Pie handles for you}

  \begin{enumerate}\small
  \item Resource allocation (e.g. starting new instances on
    ScienceCloud)
  \item Selection of resources for each application in the session
  \item Data transfer (e.g. copying input files in the new instances)
  \item Remote execution of the application
  \item Retrieval of results (e.g. copying output files from the
    running instance)
  \item De-allocation of resources
  \end{enumerate}

\end{frame}


\part{Concepts and glossary}

\begin{frame}
  \frametitle{GC3Pie glossary: Application}
  \begin{quote}
    GC3Pie runs \alert<2-3>{user applications}
    \\
    on clusters and IaaS cloud resources
  \end{quote}

  \uncover<2-3>{
    \+ \alert<2>{An \texttt{Application} is just a command to execute.}

    \+
    \only<2>{If you can run it in the terminal, \\ you can run it in GC3Pie.}
    \only<3>{
      \alert<3>{A single execution of an \texttt{Application} \\ is indeed called a \texttt{Run}.}

      \+ (Other systems might call this a ``job''.)
    }
  }
\end{frame}


\begin{frame}
  \frametitle{GC3Pie glossary: Task}
  \begin{quote}
    GC3Pie \alert{runs} user applications
    \\
    on clusters and IaaS cloud resources
  \end{quote}

  \+ More generally, GC3Pie runs \texttt{Task}s.

  \+ \texttt{Task}s are a superset of applications,
  \\ in that they include workflows.
\end{frame}


\begin{frame}
  \frametitle{GC3Pie glossary: Resources}
  \begin{quote}
    GC3Pie runs user applications
    \\
    on clusters and IaaS cloud \alert{resources}
  \end{quote}

  \+ \alert{\texttt{Resource}s are the computing infrastructures \\ where GC3Pie executes applications.}

  \+ Resources may include: your laptop, the ``Ubelix'' cluster, SWITCHengines, Amazon EC2.
\end{frame}


\part{Workflow scaffolding}

\begin{frame}[fragile]
  \frametitle{Let's start coding!}
  \begin{columns}
    \begin{column}{0.7\linewidth}
\begin{python}
from gc3libs.cmdline \
  import SessionBasedScript

if __name__ == '__main__':
  import ex10a
  ex10a.AScript().run()

class AScript(SessionBasedScript):
  """
  Minimal workflow scaffolding.
  """
  def __init__(self):
    super(AScript, self).__init__(
        version='1.0')
  def new_tasks(self, extra):
    apps_to_run = [ ]
    return apps_to_run
\end{python}
    \end{column}
    \begin{column}{0.3\linewidth}
      \href{https://raw.githubusercontent.com/riccardomurri/python-for-science-intro/master/download/ex10a.py}{Download} this code into a file named \texttt{ex10a.py}

      \+
      Open it in your favorite text editor.
    \end{column}
  \end{columns}
\end{frame}


\begin{frame}[fragile]
  \begin{exercise*}[10.A]

    \+
    \href{https://raw.githubusercontent.com/riccardomurri/python-for-science-intro/master/download/ex10a.py}{Download this code} into a file named \texttt{ex10a.py}

    \begin{enumerate}
    \item Run the following command:
\begin{sh}
$ python ex10a.py --help
\end{sh}%$
        Where does the program description in the help text come from?
        Is there anything weird in other parts of the help text?

    \item Run the following command:
\begin{sh}
$ python ex10a.py
\end{sh}%$
        What happens?
      \end{enumerate}
  \end{exercise*}
\end{frame}


\begin{frame}[fragile]
  \begin{columns}[t]
    \begin{column}{0.7\linewidth}
\begin{python}
from gc3libs.cmdline \
  import SessionBasedScript

~\HL{if \_\_name\_\_ == '\_\_main\_\_':}~
  ~\HL{import ex10a}~
  ~\HL{ex10a.AScript().run()}~

class AScript(SessionBasedScript):
  """
  Minimal workflow scaffolding.
  """
  def __init__(self):
    super(AScript, self).__init__(
        version='1.0')
  def new_tasks(self, extra):
    apps_to_run = [ ]
    return apps_to_run
\end{python}
    \end{column}
    \begin{column}{0.4\linewidth}
      \begin{flushright}
        These lines are needed in every session-based script.

        \+
        See \href{https://github.com/uzh/gc3pie/issues/95}{issue 95} for details.
      \end{flushright}
    \end{column}
  \end{columns}
\end{frame}


\begin{frame}[fragile]
  \begin{columns}[t]
    \begin{column}{0.7\linewidth}
\begin{python}
from gc3libs.cmdline \
  import SessionBasedScript

if __name__ == '__main__':
  import ~\HL{ex10a}~
  ~\HL{ex10a}~.AScript().run()

class AScript(SessionBasedScript):
  """
  Minimal workflow scaffolding.
  """
  def __init__(self):
    super(AScript, self).__init__(
        version='1.0')
  def new_tasks(self, extra):
    apps_to_run = [ ]
    return apps_to_run
\end{python}
    \end{column}
    \begin{column}{0.4\linewidth}
      \begin{flushright}
        For this to work, it is \textbf{needed} that this is the
        actual file name.
      \end{flushright}
    \end{column}
  \end{columns}
\end{frame}


\begin{frame}[fragile]
  \begin{columns}
    \begin{column}{0.7\linewidth}
\begin{python}
from gc3libs.cmdline \
  import SessionBasedScript

if __name__ == '__main__':
  import ex10a
  ex10a.AScript().run()

class AScript(SessionBasedScript):
  """
  ~\HL{Minimal workflow scaffolding.}~
  """
  def __init__(self):
    super(AScript, self).__init__(
        version='1.0')
  def new_tasks(self, extra):
    apps_to_run = [ ]
    return apps_to_run
\end{python}
    \end{column}
    \begin{column}{0.4\linewidth}
      \begin{flushright}
        This is the \\ program's help text!
      \end{flushright}
    \end{column}
  \end{columns}
\end{frame}


\begin{frame}[fragile]
  \begin{columns}
    \begin{column}{0.7\linewidth}
\begin{python}
from gc3libs.cmdline \
  import SessionBasedScript

if __name__ == '__main__':
  import ex10a
  ex10a.AScript().run()

class AScript(SessionBasedScript):
  """
  Minimal workflow scaffolding.
  """
  def __init__(self):
    super(AScript, self).__init__(
      ~\HL{version='1.0'}~)
  def new_tasks(self, extra):
    apps_to_run = [ ]
    return apps_to_run
\end{python}
    \end{column}
    \begin{column}{0.4\linewidth}
      \begin{flushright}
        A version number \\ is \textbf{mandatory}.
      \end{flushright}
    \end{column}
  \end{columns}
\end{frame}


\begin{frame}[fragile]
  \begin{columns}
    \begin{column}{0.7\linewidth}
\begin{python}
from gc3libs.cmdline \
  import SessionBasedScript

if __name__ == '__main__':
  import ex10a
  ex10a.AScript().run()

class AScript(SessionBasedScript):
  """
  Minimal workflow scaffolding.
  """
  def __init__(self):
    super(AScript, self).__init__(
        version='1.0')
  def new_tasks(self, extra):
    ~\HL{apps\_to\_run = [ ]}~
    ~\HL{return apps\_to\_run}~
\end{python}
    \end{column}
    \begin{column}{0.4\linewidth}
      \begin{flushright}
        \textbf{This is the core of the script.}

        \+
        Return a list of \texttt{Application} objects, that GC3Pie will execute.
      \end{flushright}
    \end{column}
  \end{columns}
\end{frame}


\section{Applications}
\part{The \texttt{Application} object}

\begin{frame}
  \frametitle{Specifying commands to run, I}

  You need to ``describe'' an application to GC3Pie, in order for
  GC3Pie to use it.

  \+
  This ``description'' is a blueprint from which many actual
  command instances can be created.

  \+
  (A few such ``descriptions'' are already part of the core library.)
\end{frame}


\begin{frame}
  \frametitle{GC3Pie application model}

  In GC3Pie, an application ``description'' is an object of the
  \texttt{gc3libs.Application} class (or subclasses thereof).

  \+
  At a minimum: provide application-specific command-line invocation.

  \+
  Advanced users can customize pre- and post-processing, react on
  state transitions, set computational requirements based on input
  files, influence scheduling.  (This is standard OOP: subclass and
  override a method.)
\end{frame}


\begin{frame}
  \frametitle{Detour: asset pricing, I}
  \small The script
  \href{https://raw.githubusercontent.com/riccardomurri/python-for-science-intro/master/download/simAsset.R}{\texttt{simAsset.R}}
  simulates asset pricing over a certain amount of time. Different pricing paths
  are generated using a \href{https://en.wikipedia.org/wiki/Wiener_process}{1D
    Brownian motion}, all starting from the same initial price.
  \begin{center}
    \includegraphics[width=0.75\linewidth]{fig/simAsset.pdf}
  \end{center}
\end{frame}


\begin{frame}[fragile]
  \frametitle{Detour: asset pricing, II}
  \small You can run the \texttt{simAsset.R} script by giving it one single
  input file in CSV format.  For example:
\begin{semiverbatim}
  \$ Rscript simAsset.R simAsset.dat
\end{semiverbatim}

  \+ Each run of \texttt{simAsset.R} script produces two output files \emph{per
    each input row}:
  \begin{description}
  \item[result$K$.csv] table of generated data: each column is a simulation path, each row is a time step;
  \item[result$K$.pdf] plot of the above.
  \end{description}

  \+ \footnotesize
  {\em (You can download a sample ``\texttt{simAsset1.dat}'' input file from
    \href{https://raw.githubusercontent.com/riccardomurri/python-for-science-intro/master/download/}{this
      URL}.)}
\end{frame}


\begin{frame}[fragile]
  \frametitle{Detour: asset pricing, III}

  Each line of the input file for \texttt{simAsset.R} contains the following
  fields, separated by a comma (`\texttt{,}'):
  \begin{description}
  \item[$S_0$] stock price today (e.g., 50)
  \item[$\mu$] expected return (e.g., 0.04)
  \item[$\sigma$] volatility (e.g., 0.1)
  \item[$\delta$] size of time steps (e.g., 0.273)
  \item[$e$] days to expiry (e.g., 1000)
  \item[$N$] number of simulation paths to generate
  \end{description}

  \+ \footnotesize
  {\em (You can download a sample ``\texttt{simAsset1.dat}'' input file from
    \href{https://raw.githubusercontent.com/riccardomurri/python-for-science-intro/master/download/}{this
      URL}.)}
\end{frame}


\begin{frame}[fragile]
\frametitle{Running the asset pricing example, I}

  Here is how you would tell GC3Pie \\ to run that command-line.

\begin{python}
from gc3libs import Application

class PricingApp1(Application):
  """Simulate asset pricing."""
  def __init__(self, infile):
    inp = basename(infile)
    Application.__init__(
      self,
      arguments=[
        'Rscript', 'simAsset.R', inp],
      inputs=['simAsset.R', infile],
      outputs=['result1.csv', 'result1.pdf'],
      output_dir='pricing1.d',
      stdout='stdout.txt')
\end{python}
\end{frame}


\begin{frame}[fragile]
\frametitle{Always inherit from Application}

  Your application class must inherit from class \texttt{gc3libs.Application}
  \+
\begin{python}
~\HL{from gc3libs import Application}~

class PricingApp1~\HL{(Application)}~:
  """Simulate asset pricing."""
  def __init__(self, infile):
    inp = basename(infile)
    Application.__init__(
      self,
      arguments=[
        'Rscript', 'simAsset.R', inp],
      inputs=['simAsset.R', infile],
      outputs=['result1.csv', 'result1.pdf'],
      output_dir='pricing1.d',
      stdout='stdout.txt')
\end{python}
\end{frame}


\begin{frame}[fragile]
  \frametitle{The \texttt{arguments} parameter, I}

  The \texttt{arguments=} parameter is the actual command-line to be invoked.

  \+
\begin{python}
class PricingApp1(Application):
  """Simulate asset pricing."""
  def __init__(self, infile):
    inp = basename(infile)
    Application.__init__(
      self,
      ~\HL{arguments=[}~
        ~\HL{"Rscript", "simAsset.R", inp],}~
      inputs=['simAsset.R', infile],
      outputs=['result1.csv', 'result1.pdf'],
      output_dir='pricing1.d',
      stdout='stdout.txt')
\end{python}
\end{frame}

\begin{frame}[fragile]
\frametitle{The \texttt{arguments} parameter, II}

The first item in the \texttt{arguments} list is the name or path to the command to run.

  \+
\begin{python}
class PricingApp1(Application):
  """Simulate asset pricing."""
  def __init__(self, infile):
    inp = basename(infile)
    Application.__init__(
      self,
      arguments=[
        ~\HL{'Rscript'}~, 'simAsset.R', inp],
      inputs=['simAsset.R', infile],
      outputs=['result1.csv', 'result1.pdf'],
      output_dir='pricing1.d',
      stdout='stdout.txt')
\end{python}
\end{frame}

\begin{frame}[fragile]
\frametitle{The \texttt{arguments} parameter, III}

The rest of the list are arguments to the program, as you would type
them at the shell prompt.

  \+
\begin{python}
class PricingApp1(Application):
  """Simulate asset pricing."""
  def __init__(self, infile):
    inp = basename(infile)
    Application.__init__(
      self,
      arguments=[
        'Rscript', ~\HL{'simAsset.R', inp}~],
      inputs=['simAsset.R', infile],
      outputs=['result1.csv', 'result1.pdf'],
      output_dir='pricing1.d',
      stdout='stdout.txt')
\end{python}
\end{frame}


\begin{frame}[fragile]
\frametitle{The \texttt{inputs} parameter, I}

The \texttt{inputs} parameter holds a list of files that you want to
\emph{copy} to the location where the command is executed. (Remember:
this might be a remote computer!)

  \+
\begin{python}
class PricingApp1(Application):
  """Simulate asset pricing."""
  def __init__(self, infile):
    inp = basename(infile)
    Application.__init__(
      self,
      arguments=[
        'Rscript', 'simAsset.R', inp],
      ~\HL{inputs=['simAsset.R', inp]}~,
      outputs=['result1.csv', 'result1.pdf'],
      output_dir='pricing1.d',
      stdout='stdout.txt')
\end{python}
\end{frame}


\begin{frame}[fragile]
  \frametitle{The \texttt{inputs} parameter, II}

  Input files retain their name during the copy, \\ but not the entire path.

  \+
  For example:
  \begin{python}
    inputs = [
      '/home/rmurri/db01/values.dat',
      '/home/rmurri/stats.csv',
    ]
  \end{python}
  will make files \emph{values.dat} and \emph{stats.csv} available in
  the command execution directory.

\end{frame}


\begin{frame}[fragile]
  \frametitle{The \texttt{inputs} parameter, III}

  You need to pass the full path name into the
  \texttt{inputs} list, but use only the ``base name'' in the command
  invocation.

\begin{python}
class PricingApp1(Application):
  """Simulate asset pricing."""
  def __init__(self, infile):
    ~\HL{inp = basename(infile)}~
    Application.__init__(
      self,
      arguments=[
        'Rscript', 'simAsset.R', ~\HL{inp}~],
      inputs=['simAsset.R', ~\HL{infile}~],
      outputs=['result1.csv', 'result1.pdf'],
      output_dir='pricing1.d',
      stdout='stdout.txt')
\end{python}
\end{frame}


\begin{frame}[fragile]
\frametitle{The \texttt{outputs} parameter, I}

The \texttt{outputs} argument list files that should be copied from
the command execution directory back to your computer.

  \+
\begin{python}
class PricingApp1(Application):
  """Simulate asset pricing."""
  def __init__(self, infile):
    inp = basename(infile)
    Application.__init__(
      self,
      arguments=[
        'Rscript', 'simAsset.R', inp],
      inputs=['simAsset.R', infile],
      ~\HL{outputs=['result1.csv', 'result1.pdf']}~,
      output_dir='pricing1.d',
      stdout='stdout.txt')
\end{python}
\end{frame}


\begin{frame}[fragile]
  \frametitle{The \texttt{outputs} parameter, II}

  Output file names are \emph{relative to the execution directory}.
  For example:
  \begin{python}
    outputs = ['result.dat', 'program.log']
  \end{python}

  \+
  (Contrast with input files, which must be specified by
  \emph{absolute path}, e.g., \texttt{/home/rmurri/values.dat})

  \+
  Any file with the given name that is found in the execution
  directory will be copied back. (\emph{Where?} See next slides!)

  \+
  If an output file is \emph{not} found, this is \emph{not} an
  error. In other words, \textbf{output files are optional}.
\end{frame}


\begin{frame}[fragile]
\frametitle{The \texttt{output\_dir} parameter, I}

The \lstinline|output_dir| parameter specifies where output filess
will be downloaded.

\+
\begin{python}
class PricingApp1(Application):
  """Simulate asset pricing."""
  def __init__(self, infile):
    inp = basename(infile)
    Application.__init__(
      self,
      arguments=[
        'Rscript', 'simAsset.R', inp],
      inputs=['simAsset.R', infile],
      outputs=['result1.csv', 'result1.pdf'],
      ~\HL{output\_dir='pricing1.d'}~,
      stdout='stdout.txt')
\end{python}
\end{frame}


\begin{frame}[fragile]
  \frametitle{The \emph{output\_dir} parameter, II}

  By default, GC3Pie does not overwrite an existing output directory:
  it will move the existing one to a backup name.

  \+
  So, if \texttt{pricing1.d} already exists, GC3Pie will:
  \begin{enumerate}
  \item rename it to \lstinline|pricing1.d.~1~|
  \item create a new directory \texttt{pricing1.d}
  \item download output files into the new directory
  \end{enumerate}
\end{frame}


\begin{frame}[fragile]
\frametitle{The \texttt{stdout} parameter}

This specifies that the command's \emph{standard output} should be
saved into a file named \texttt{stdout.txt} and retrieved along with
the other output files.

  \+
\begin{python}
class PricingApp1(Application):
  """Simulate asset pricing."""
  def __init__(self, infile):
    inp = basename(infile)
    Application.__init__(
      self,
      arguments=[
        'Rscript', 'simAsset.R', inp],
      inputs=['simAsset.R', infile],
      outputs=['result1.csv', 'result1.pdf'],
      output_dir='pricing1.d',
      ~\HL{stdout="stdout.txt"}~)
\end{python}
\end{frame}


\begin{frame}[fragile]
\frametitle{(The \texttt{stderr} parameter)}

There's a corresponding \texttt{stderr} option for the command's
\emph{standard error} stream.

\begin{python}
class PricingApp1(Application):
  """Simulate asset pricing."""
  def __init__(self, infile):
    inp = basename(infile)
    Application.__init__(
      self,
      arguments=[
        'Rscript', 'simAsset.R', inp],
      inputs=['simAsset.R', infile],
      outputs=['result1.csv', 'result1.pdf'],
      output_dir='pricing1.d',
      stdout='stdout.txt')
      ~\HL{stderr="stderr.txt"}~)
\end{python}
\end{frame}


\begin{frame}
  \frametitle{Mixing \texttt{stdout} and \texttt{stderr} capture}

  You can specify \textbf{either one} of the \texttt{stdout} and
  \texttt{stderr} parameters, \textbf{or both}.

  \+
  If you give both, and they have the same value, then
  \texttt{stdout} and \texttt{stderr} will be intermixed just as they
  are in normal screen output.
\end{frame}


\begin{frame}[fragile]
  \frametitle{Let's run!}
  In order for a session-based script to execute something, its
  \texttt{new\_tasks()} method must return a list of
  \texttt{Application} objects to run.

\+
\begin{python}
class AScript(SessionBasedScript):
  # ...
  def new_tasks(self, extra):
    # `self.param.args' is the list
    # of command-line arguments
    input_file = self.params.args[0]
    app = PricingApp1(input_file)
    return [app]
\end{python}
\end{frame}


\begin{frame}[fragile]
  \small

  \begin{exercise*}[10.B]

    \+
    Edit the \texttt{ex10a.py} file: insert the code to define the
    \href{https://raw.githubusercontent.com/riccardomurri/python-for-science-intro/master/downloads/pricingapp1.py}{\texttt{PricingApp1}}
    application, and modify the \texttt{new\_tasks()} method to return
    one instance of it (as in the previous slide).

    \+
    Can you process the \href{https://raw.githubusercontent.com/riccardomurri/python-for-science-intro/master/fig/simAsset1.dat}{\texttt{simAsset1.dat}} file using this GC3Pie script?

    \+ \footnotesize
    {\em (You can download the code for \texttt{PricingApp1} and the ``\texttt{simAsset1.dat}'' file from
   \href{https://raw.githubusercontent.com/riccardomurri/python-for-science-intro/master/download/}{this
      URL}.)}
  \end{exercise*}

  \+
  \begin{exercise*}[10.C]
    What happens if you omit the \texttt{result1.pdf} file from the
    \texttt{outputs=[...]} line?  And if you omit the \texttt{result1.csv} one?
  \end{exercise*}
\end{frame}


\begin{frame}[fragile]
  \small

  \begin{exercise*}[10.D]

    Edit the script from Exercise 10.B above and add the ability to process
    multiple input lines: for each row of the file given on the command line, an
    instance of \texttt{ProcessingApp1} should be run. Test this with the
    \href{https://raw.githubusercontent.com/riccardomurri/python-for-science-intro/master/fig/simAsset2.dat}{\texttt{simAsset2.dat}}
  \end{exercise*}

  \+
  \begin{exercise*}[10.E]
    Edit the script from Exercise 10.B above and add the ability to
    process multiple input files: for each file name given on the command
    line, an instance of \texttt{ProcessingApp1} should be run.
  \end{exercise*}
\end{frame}


\section{Resource definition}
\part{Resource definition}

\begin{frame}[fragile]
  \frametitle{The \texttt{gservers} command}

  The \texttt{gservers} command is used to see \alert<2>{configured} and
  available resources.

\+
\begin{stdout}
$ gservers
+---------------------+--------------------------+-----------+
|                     | localhost                |           |
+---------------------+--------------------------+-----------+
|            frontend | ( Frontend host name )   | localhost |
|                type | ( Access mode )          | shellcmd  |
|             updated | ( Accessible? )          | True      |
|              queued | ( Total queued jobs )    | 0         |
|         user_queued | ( Own queued jobs )      | 0         |
|            user_run | ( Own running jobs )     | 6         |
|   max_cores_per_job | ( Max cores per job )    | 4         |
| max_memory_per_core | ( Max memory per core )  | 8GiB      |
|        max_walltime | ( Max walltime per job ) | 8hour     |
+---------------------+--------------------------+-----------+
\end{stdout}%$

\uncover<2>{%
  \small \alert<2>{Resources are defined in file \texttt{\$HOME/.gc3/gc3pie.conf}}
}
\end{frame}


\begin{frame}[fragile,label=resources]
  \frametitle{Example execution resources: local host}
  \begin{columns}[t]
    \begin{column}{0.5\textwidth}
      Allow GC3Pie to run tasks on the local computer.

      \+ This is the default installed by GC3Pie
      into \lstinline|$HOME/.gc3/gc3pie.conf| %$
    \end{column}
    \begin{column}{0.5\textwidth}
  \begin{stdout}
[resource/localhost]
enabled = yes
type = shellcmd
frontend = localhost
transport = local
max_cores_per_job = 2
max_memory_per_core = 2GiB
max_walltime = 8 hours
max_cores = 2
architecture = x86_64
auth = none
override = no
\end{stdout}
    \end{column}
  \end{columns}
\end{frame}


\begin{frame}[fragile]
  \frametitle{Example execution resources: SLURM}
  \begin{columns}[t]
    \begin{column}{0.5\textwidth}
      Allow submission of jobs to the ``Hydra'' cluster.
    \end{column}
    \begin{column}{0.5\textwidth}
\begin{stdout}
[resource/hydra]
enabled = no
type = slurm
frontend = login.s3it.uzh.ch
transport = ssh
auth = ssh_user_rmurri
max_walltime = 1 day
max_cores = 96
max_cores_per_job = 64
max_memory_per_core = 1 TiB
architecture = x86_64
prologue_content =
  module load cluster/largemem

[auth/ssh_user_rmurri]
type=ssh
username=rmurri
\end{stdout}
    \end{column}
  \end{columns}
\end{frame}


\begin{frame}[fragile]
  \frametitle{Example execution resources: OpenStack}
  \begin{columns}[t]
    \begin{column}{0.5\textwidth}
\begin{stdout}
[resource/sciencecloud]
enabled=no
type=openstack+shellcmd
auth=openstack

vm_pool_max_size = 32
security_group_name=default
security_group_rules=
  tcp:22:22:0.0.0.0/0,
  icmp:-1:-1:0.0.0.0/0
network_ids=
  c86b320c-9542-4032-a951-c8a068894cc2

# definition of a single execution VM
instance_type=1cpu-4ram-hpc
image_id=2b227d15-8f6a-42b0-b744-ede52ebe59f7

max_cores_per_job = 8
max_memory_per_core = 4 GiB
max_walltime = 90 days
max_cores = 32
architecture = x86_64

# how to connect
vm_auth=ssh_user_ubuntu
keypair_name=rmurri
public_key=~/.ssh/id_dsa.pub
\end{stdout}
    \end{column}
    \begin{column}{0.5\textwidth}
      \begin{stdout}
[auth/ssh_user_ubuntu]
# default user on Ubuntu VM images
type=ssh
username=ubuntu


[auth/openstack]
# only need to set the `type` here;
# any other value will be taken from
# the `OS\_*` environment variables
type = openstack
      \end{stdout}

      \+\+\+
      Allow running tasks on the ``ScienceCloud'' VM infrastructure.

      % \+ \footnotesize
      % Cloud-based submission happens in two steps:
      % \emph{(1)} create a VM, \emph{(2)} SSH to it and run jobs.  Each
      % step requires different connection and authentication.
    \end{column}
  \end{columns}
\end{frame}


\begin{frame}
  \frametitle{Select execution resource}

  When multiple resources are available, you can select where
  applications will be run with option \texttt{--resource}/\texttt{-r}:
\begin{semiverbatim}
    \$ ./script.py -r localhost
\end{semiverbatim}

  \+ The resource name must exists in the configuration file (i.e.,
  check \texttt{gservers}' output).

  \+ Stopping a script and re-starting it with a different resource
  will likely result in an error: old tasks can no longer be found.
\end{frame}


\part{Post-processing}

\begin{frame}[fragile]
  \frametitle{Post-processing features, I}

  When the remote computation is done, the \texttt{terminated} method
  of the application instance is called.

  \+
  The path to the output directory is available as
  \lstinline|self.output_dir|.

  \+
  If \texttt{stdout} and \texttt{stderr}
  have been captured, the \textbf{relative} paths to the capture files
  are available as \lstinline|self.stdout| and
  \lstinline|self.stderr|.
\end{frame}


\begin{frame}[fragile]
  \frametitle{Post-processing features, II}

  For example, the following code logs a warning message if the
  standard error output is non-empty:
\begin{python}
class MyApp(Application):
  # ...
  def terminated(self):
    stderr_file = self.output_dir+"/"+self.stderr
    stderr_size = os.stat(stderr_file).st_size
    if stderr_size > 0:
      gc3libs.log.warn(
        "Application %s reported errors!", self)
\end{python}
\end{frame}


\begin{frame}[fragile]
  \frametitle{Useful in post-processing}\small

  These attributes are available in the \texttt{terminated()} method:

  \+
  \begin{describe}{\lstinline|self.inputs|}
    Python dictionary, mapping local (absolute) paths to remote paths (relative
    to execution directory)
  \end{describe}

  \+
  \begin{describe}{\lstinline|self.outputs|}
    Python dictionary, mapping remote paths (relative to execution directory) to
    \emph{URLs} where they have been copied. In particular,
    \lstinline|self.outputs.keys()| is the list of output file names.
  \end{describe}

  \+
  \begin{describe}{\lstinline|self.output_dir|}
    Path to the local directory where output files have been downloaded.
  \end{describe}
\end{frame}


\begin{frame}
  \begin{exercise*}[10.F]
    Modify class \texttt{PricingApp1} so that, when the task is done, the
    \texttt{result1.csv} file is read and the average final price is computed
    and printed to the screen together with the initial price.

    \+ You might need to check for the existence of the output file, and that it
    contains the correct output: not all tasks are successful!
  \end{exercise*}

  \begin{exercise*}[10.G]
    Bonus points if you can, instead of printing the initial and final prices,
    save them to a file \texttt{pricing\_totals.csv}.
  \end{exercise*}
\end{frame}


\begin{frame}
  \frametitle{Global post-processing, I}
In order to do ``global'' post-processing (i.e., across \texttt{Application}
objects), you need to hook into the GC3Pie script's main loop:

\+
\begin{describe}{\lstinline|after_main_loop(self)|}
  to execute some code \emph{after} the main loop, i.e., before the script
  quits. A list of all Application objects is available in the
  \lstinline|self.session.tasks.values()| list.
\end{describe}
\end{frame}


\begin{frame}[fragile]
  \frametitle{Global post-processing, II}
  Example: compute statistical distribution of termination statuses:

  \begin{python}
def after_main_loop(self):
  # check that all tasks are terminated
  can_postprocess = True
  for task in self.session.tasks.values():
    if task.execution.state != 'TERMINATED':
      can_postprocess = False
      break
  if can_postprocess:
    # do stuff... (see next slide)
  \end{python}
\end{frame}


\begin{frame}[fragile]
  \frametitle{Global post-processing, III}
  Example: compute statistical distribution of termination statuses (cont'd):

  \begin{python}
def after_main_loop(self):
  # ... (see prev slide)
  if can_postprocess:
    status_counts = defaultdict(int)
    for app in self.session.tasks.values():
      termstatus = app.execution.returncode
      status_counts[termstatus] += 1
  \end{python}

  \+\small Variable \lstinline|self.session.tasks| holds a mapping
  \lstinline|JobID|~$\Rightarrow$~\lstinline|Application|; thus
  \lstinline|self.session.tasks.values()| is a list of all the
  \texttt{Application} instances returned by \lstinline|new_tasks|
\end{frame}


\begin{frame}
  \begin{exercise*}[10.H]
    Modify the script from the last exercise to produce a scatter plot of the
    initial and average final prices as a final post-processing step when
    everything else is done.
  \end{exercise*}
\end{frame}


\part{Application requirements}

\begin{frame}
  Applications need to allocate computing resources.
  For instance: request 4 processors for 8 hours.

  \+
  GC3Pie allows requesting:
  \begin{itemize}
  \item the number of processors that a job can use,
  \item the architecture (32-bit or 64-bit) of these processors,
  \item the guaranteed duration of a job,
  \item the amount of memory that a job can use (per processor).
  \end{itemize}

  \+
  More fine-grained matching is possible, but outside the scope of
  this introductory training.
\end{frame}


\begin{frame}[fragile]
  Resources are requested using additional constructor parameters for
  \texttt{Application} objects.

  \+
  The allowed parameters are:
  \lstinline|requested_cores|,
  \lstinline|requested_architecture|,
  \lstinline|requested_walltime|,
  \lstinline|requested_memory|.
\end{frame}


\begin{frame}[fragile]
  \frametitle{Running parallel jobs}

  You request allocation of a certain number of processors using the
  \lstinline|requested_cores| parameter: set it to the number of CPU
  cores that you want.

  \+
  For example, the following runs the command \texttt{mpixexec
    simulator} on 4 processors:
  \begin{python}
  class ZodsApplication(Application):
    # ...
    Application.__init__(self,
      arguments=['mpiexec', '-n', '4', 'simulator'],
      # ...
      ~\HL{requested\_cores=4}~)
  \end{python}

  \+
  {\small Note that GC3Pie only guarantees the availability of a certain
    number of processors; it is your application's responsibility to use
    them, e.g., by starting a command using MPI or any other parallel
    processing mechanism.}
\end{frame}


\begin{frame}[fragile]
  \frametitle{Requesting processor architecture}

  If you send the compiled executable along with your application, you
  need to select only resources that can run that binary file.

  \+
  The \lstinline|requested_architecture| parameter provides the
  choice between \lstinline|gc3libs.Run.Arch.X86_64| (for 64-bit
  Intel/AMD computers) and \lstinline|gc3libs.Run.Arch.X86_32| (for
  32-bit ones):
  \begin{python}
  from gc3libs import Run
  class CodemlApplication(Application):
    # ...
    Application.__init__(self,
      arguments=['./codeml.bin'],
      inputs = ['/usr/local/bin/codeml', ...]
      # ...
      ~\HL{requested\_architecture=Run.Arch.X86\_64}~)
  \end{python}
\end{frame}


\begin{frame}[fragile]
  \frametitle{Requesting running time}

  In order to ensure that your job is allotted enough time to run on
  the remote computing system, use the \lstinline|requested_walltime|
  parameter.

  \+
  \begin{python}
  ~\HL{\textbf{from} gc3libs.quantity}~ \
    ~\HL{\textbf{import} minutes, hours, days}~

  class CodemlApplication(Application):
    # ...
    Application.__init__(self,
      # ...
      ~\HL{requested\_walltime=8*hours}~)
  \end{python}

  \+
  You \textbf{must} use a \texttt{gc3libs.quantity} multiple for the
  \lstinline|requested_walltime| parameter; any other value will be
  rejected with an error.
\end{frame}

\begin{frame}[fragile]
  \frametitle{Units of time}
  The Python module \texttt{gc3libs.quantity} provides units for
  expressing time requirements in days, hours, minutes, seconds.

  \+
  Just multiply the unit by the amount you need:
  \begin{python}
    >>> an_hour = 1*hours
  \end{python}
  Or sum the amounts:
  \begin{python}
    >>> two_days = 1*days + 24*hours
  \end{python}

  \+
  GC3Pie will automatically perform the conversions:
  \begin{python}
    >>> two_hours = 2*hours
    >>> another_two_hours = 7200*seconds
    >>> two_hours == another_two_hours
    True
  \end{python}
\end{frame}


\begin{frame}[fragile]
  \frametitle{Requesting memory}
  In order to secure a certain amount of memory for a job, use the
  \lstinline|requested_memory| parameter.

  \+
  Example:
\begin{python}
  ~\HL{\textbf{from} gc3libs.quantity \textbf{import} GB, MB, kB}~
  class CodemlApplication(Application):
    # ...
    Application.__init__(self,
      # ...
      ~\HL{requested\_memory=8*GB}~)
\end{python}

  \+
  Note that \lstinline|requested_memory| expresses the total
  memory used by the job!
\end{frame}

\begin{frame}[fragile]
  \frametitle{Units of memory}
  The Python module \texttt{gc3libs.quantity} provides units for
  expressing memory requirements in kilo-, Mega- and Giga-bytes.

  \+
  Just multiply the unit by the amount you need:
\begin{python}
  >>> a_gigabyte = 1*GB
  >>> two_megabytes = 2*MB
\end{python}

  \+
  GC3Pie will automatically perform the conversions:
  \begin{python}
    >>> two_gigabytes = 2*GB
    >>> another_two_gbs = 2000*MB
    >>> two_gigabytes == another_two_gbs
    True
  \end{python}
\end{frame}


\begin{frame}[fragile]
  \frametitle{All together now}

\begin{python}
from gc3libs.quantity import GB, MB, kB
from gc3libs.quantity import days, hours, minutes

class CodemlApplication(Application):
  # ...
  Application.__init__(self,
    # ...
    requested_cores=1,
    requested_memory=2*GB,
    requested_walltime=8*hours)
\end{python}

  \+ When several resource requirements are specified, GC3Pie tries to
  satisfy \emph{all} of them.  \textbf{If this is not possible, task
  submission fails and the task stays in state \emph{NEW}.}

\end{frame}


\end{document}

%%% Local Variables:
%%% mode: latex
%%% TeX-master: t
%%% End:
